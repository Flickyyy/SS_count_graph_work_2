\section{Задание 3. ДУ второго порядка}

\textbf{Условие.}

Пружинный маятник движется по закону:

\[y^{\prime\prime} + p(t)y^\prime + q(t)y = f(t)\]

\begin{enumerate}
    \item Запишите однородное уравнение движения маятника. Выясните, почему движение описывается уравнением такого вида (каков физический смысл коэффициентовлевой части уравнения).
    \item Установите характер движения (периодический, апериодический) при данных $p(t)$ и $q(t)$.
    \item Найдите ФСР ЛОДУ и убедитесь в ее линейной независимости с помощью вронскиана.
    \item Найдите общее решение ЛОДУ.
    \item Задайте начальные условия в момент $t_0 = 0$ и найдите удовлетворяющее им частное решение ЛОДУ. Изобразите закон движения в системе координат.
    \item Составьте линейное неоднородное дифференциальное уравнение (ЛНДУ) с правой частью $f(t)$. Выясните физический смысл функции $f(t)$.
    \item Найдите решение ЛНДУ, удовлетворяющее начальным условиям. Изобразите закон движения в системе координат.
    \item Сделайте вывод о влиянии на движение функции $f(t)$.
\end{enumerate}

\[p(t) = 4, q(t) = 5, f(t) = t^2 e^{2t}\]

\vspace{10mm}
\textbf{Решение.}

It is empty but you can fill it!

\textit{Ответ}: It is empty but you can fill it!

